%!TEX root = ../dissertation.tex
% the abstract

The explosion of data arising from advances in high throughput sequencing has allowed scientists to study genomics in far greater detail.  However, this high resolution picture of cells often makes it difficult to see the higher level functions, features and structure in the biology that lead to phenotypic outcomes.  With these advancements come new challenges in genomic analysis. The biological complexity of sample is not fully captured by a single snap shot of the sample's genome or gene expression profile. It would be simple if all phenotypes could be explained with a simple additive model consisting of the measured gene expression or variants in an organism.  However, it has become increasingly clear that the drivers of fundamental biological processes involve complex systems of interactions between numerous genomic components.  These interactions are often described using network models, such as gene regulatory models.  Understanding how these networks should be constructed is an active field involving many types of interpretations of edges and methods for inference. However, relatively few methods exist for identifying network changes between phenotypic states or experimental conditions.  In chapter 2, we address this problem by proposing a method for estimating transcription factors that characterize changes in these networks.

In network inference we are seeking to find genes with related expression patterns. However, unknown and unmeasured structure is known to exists in genomic studies and has the potential to bias associations by confounding the relationship between features and phenotypes. Substantial work has already been published which attempts to identify and remove the impact of this unwanted variation, but subtle effects will continue to remain. Left uncorrected, this hidden structure may cause spurious correlations in genome wide association studies and gene expression analyses. As the field advances, new applications and new technologies call for methods which improve on existing tools and utilize all available information to provide better estimates.  Two of the chapters presented here deal with problems of identifying and removing unwanted structure in data. Chapter 4 addresses a previously undocumented problem in the topic of gene coexpression by identifying and controlling for batch effect at the covariance level. In other words, we address confounding where coexpression is induced in a subset of samples by those samples' membership in a batch. We propose a method to correct the coexpression matrix that recognizes the modular nature of gene expression using a regression model for the eigenvectors of the expression correlation. Chapter 3 presents a method for identifying heterogeneity in genome studies aimed at high-throughput DNA-sequencing. This approach exploits the increased informativeness of low frequency variants to provide a higher resolution picture of population structure by more precisely measuring genetic similarity.
