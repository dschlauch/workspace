%!TEX root = ../dissertation.tex
\chapter{Introduction}
\label{introduction}
\newthought{The development of high-throughput technologies} over the last two decades has brought significant promise towards understanding molecular biology and has shed light on the genomic involvement in the progression of human disease. Microarray and sequencing technologies have allowed us to interrogate many different types of biological problems at the molecular level, including the study of the genome, the transcriptome, the epigenome and other 'omics and dramatically reduced cost. For example, RNA-Seq is commonly used to measure the abundance of RNA (often selected for mRNA) in a biological sample with the hope that relative quantities of RNA mapping to particular genes will help explain proteomic, cellular and phenotypic differences we observe at a higher level. But, as this ability to collect data has increased, so too have the statistical, biological and computational challenges that accompany questions about how healthy disease transitions to disease. It is clear that most phenotypic differentiation is not attributable to single units (genes or variants, for example) responsible for high level function. Instead, we observe that cell states are more adequately described with models that include numerous interacting features. 

We are interested in hidden structures within the data that can tell us more about biological systems than the sum of individual components of the data. However, this underlying framework may represent many types of hidden systems, depending on the particular area of study and data. In some cases, this structure is important to understanding the mechanisms which drive disease, such as when an important biological process is disrupted in disease cells. But other times that structure arises from other variables that may confound analyses, such as the presence of ancestral heterogeneity in genetic studies. It is critical to find and address these artifacts where possible, as failing to do so leads to inflated rates of false positives.

Two such examples of unwanted structure in the data are seen with batch effect in gene expression and population stratification in statistical genetics. In each case, heterogeneity of the samples creates unwanted variation in the data. These effects have been widely studied and are known to produce spurious results in analyses such as Genome Wide Association Studies (GWAS) and differential gene expression analyses. Many methods have been proposed to address these issues. Some of these approaches require the advanced knowledge of the sample variables which cause the underlying structure (such as batch effect), but others require the estimation of the sources of variation in the data. If the source of unwanted variation is unknown, it is common to estimate it by first estimating a similarity matrix across samples.  This matrix can be used with an eigendecomposition or with a linear mixed model to control the type I error. It's clear that the efficacy of these methods is directly impacted by our ability to infer genetic similarity.  With the increased use of DNA sequencing relative to DNA microarrays, we have increased resolution to infer genetic similarity.  Furthermore, it has been shown that the lower frequency variants, visible only for sequencing studies, are the most informative of ancestry, owing to their more recent average emergence in human evolution. Chapter 3 in this dissertation proposed a new method in this field which exploits this new information to generate a higher resolution picture of ancestry. This tool has wide utility in identifying subtle population structure and cryptic relatedness in studies, particularly those involving purportedly homogeneous populations, but in fact exhibit subtle structure.  

In the case of gene expression studies, we also see the presence of unwanted structure in the form of batch effect.  This problem is typically motivated by differential gene expression analyses where we are most interested in determining genes or gene sets which show relative changes in mRNA abundances across phenotypic groups. Methods such as ComBat and Surrogate Variable Analysis have been demonstrated to be very effective in this context and are widely used. More recently, scientists have become interested in developing gene networks which focus on gene coexpression rather than gene expression. In these analyses, it is not the relative abundance of a gene that we are concerned with - it is the manner in which that gene's expression pattern matches others that provide clues to its biological function. Significant work has been undertaken describing cellular states with gene networks. Simply put, researchers are interested in uncovering the manner in which genes are functionally connected. There are many ways to describe gene relationships, but often, if these models describe direct regulatory function we refer to them as Gene Regulatory Networks (GRN) and if they more generally imply ``guilt-by-association'' coexpression in an undirected graph, we call them Gene Coexpression Networks (GCN). This has lead to a growth in our understanding of biological function along with an appreciation for the complexity of molecular pathways and the biological processes that accompany them. However, the batch correction tools which are in wide use today make corrections at the individual gene level, which does not necessarily adjust for confounding by coexpression. Chapter 4 describes this problem and presents an approach that produces a model for the coexpression matrix that can be used to control for batch effect and other confounding covariates.  This tool has applications for any of the numerous methods which utilize a coexpression matrix in the analysis, which is common in gene network inference.

Gene regulatory network inference is important for understanding how transcription factors influence downstream genes, but many studies involve cases and controls and are designed to uncover the molecular mechanisms which separate the two. These investigations are less interested in the topology overall networks, but rather choose to focus the interactions that differ between states. In this context, our goals can be divided into two parts, (1) the construction of gene regulatory networks and (2) the analysis of the structural changes between those networks. Due to the complexity of the underlying networks and the high dimensionality of typical datasets, these challenges remain open problems. In this dissertation we the explore novel methods developed for gaining insight into network transformations between cases and controls in a complex disease.
Biological states are characterized by distinct patterns of gene expression that reflect each phenotype's active cellular processes. Driving these phenotypes are GRNs for which transcriptions factors control when and to what degree individual genes are expressed. Phenotypic transitions, such as those that occur when disease arises from healthy tissue, are associated with changes in these networks. In this context, we are less interested in inferring the general biological landscape, but are more interested in interrogating the transcription factor-gene relationships that change from one phenotype to another. While many methods exist for network inference, few approaches are designed for evaluating differential gene regulatory networks.  A simple approach involves simply finding the difference between two inferred networks. In Chapter 2, we present a new approach to understanding these transitions. MONSTER models phenotypic-specific regulatory networks and then estimates a ``transition matrix'' that converts one state to another. By examining the properties of the transition matrix, we can gain insight into regulatory changes associated with phenotypic state transition.
