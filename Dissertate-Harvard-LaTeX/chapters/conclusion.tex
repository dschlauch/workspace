%!TEX root = ../dissertation.tex
\chapter{Conclusion}
\label{conclusion}

For the foreseeable future, we will continue to produce greater quantities of genomic data with improved precision at faster rates and cheaper costs.  Scientists will have unprecedented access to the data that could lead to the understanding of human disease and point to potential targets for intervention.  However, as French Mathematician Henri Poincare once said, "Science is built of facts the way a house is built of bricks. But an accumulation of facts is no more science than a pile of bricks is a house." It is already clear that many of the bottlenecks in the path to understanding lie not in our ability to generate but in the analysis of that data.  

Perhaps the most significant bottleneck in genomic analysis comes from the recognition that biomolecular functions are extraordinarily complex.  With respect to finding causative genomic features of diseases, many of the low hanging fruit have long been picked. For example, the relatively small set of heritable diseases which are adequately explained by the additive effects of small number causal variants have been mostly identified.  Remaining are the vast range of complex and or rare diseases which are driven not by a single genetic risk, but by an intricate system with contributions from numerous genomic, transcriptomic, epigenomic, and environmental factors. Additionally, many of the characteristics of biological function that may be involved in disease are not even observable with single snapshots of a sample.  The measurements of interest may be the way certain genomic features interact with one another, not their isolated abundances, which cannot be estimated with a single observations. Simple models may not be appropriate for these diseases and we must therefore consider interacting elements, such as we do in gene networks, to describe the molecular mechanisms which drive cellular states. 

The growing field of personalized medicine calls for the tailoring of treatment regimens based on in part on the molecular signatures specific to an individual.  For example, genomic biomarkers such as somatic or germline mutations in cancer can suggest a patient response to a drug. Cancer in particular would benefit network based characterizations, owing to the fact that its high degree of complexity and heterogeneity makes it more appropriately described in those terms. Biomarker discovery is predicted to be heavily dependent on the ability to infer networks and understand the mechanisms that drive the change from healthy to disease.  Our work in this area, presented in chapter 2, is an important contribution to addressing this problem. In that chapter we outlined an approach to implicate certain transcription factors whose targeting pattern changes best explained the transition between cellular states. We operated under the recognition that the size and scale of the problem along with the degree of technical and biological noise in the data made it unreasonable to identify specific TF-gene interactions with a satisfactory rate of false positives. With improvements in data quality and quantity, future work will focus on specific regulatory events that are altered across experimental conditions.  This may be accelerated via the integration of complementary data sources, such ChIP-Seq and methylation sequencing, where we may gain additional independent information on gene regulation.

As the effects that we search for become smaller and more complex, greater importance lies in the ability to remove the impact of sources subtle, complex bias such as those described in these chapters. Addressing these issues with proper quality control will be critical for preventing the reporting of spurious results. QC is not simply a process stemming from imperfect data generation, which can become obsolete as the technology improves. In some cases, the unwanted artifacts arise from real biology, such as in the case of population structure. Because of this we can't rely on superior technologies to address these problems. We showed examples of this concept in chapters 3 and 4 where important structural features of the data reveal themselves when measured across samples. 

The revolutionary advancements in data generation have touched all virtually fields of research.  In addition to data creation, data access has improved as well.  Whereas previously, clinical and molecular data was frequently housed in isolated data silos controlled by separate entities, we are now seeing greater collaboration and sharing. As Dr. John Quackenbush often puts it, "Every revolution in science has been driven by one and only one thing: access to data." The methods described in these chapters depend on the availability of new data and vice versa.  This new data combined with new methods will be critical in bridging the gap from data to understanding.