%!TEX root = ../dissertation.tex
\chapter{Conclusion}
\label{conclusion}

We continue to produce greater quantities of genomic data with greater precision at faster rates and cheaper costs.  Moving forward, it is clear that many of the bottlenecks in the path to understanding lie not in our ability to generate but in the analysis of that data.  With respect to finding causative genomic features of diseases, many of the low hanging fruit have long been picked.  The relatively small set of heritable diseases which are adequately explained by the additive effects of small number causal variants have been identified.  Remaining is the vast range of complex diseases which are driven not by a single genetic risk, but by an intricate system with contributions from genomic, transcriptomic, epigenomic, and environmental factors.  Simple models may not be appropriate for these diseases and we consider interacting elements, such as we do in networks, to describe the molecular mechanisms which drive cellular states.
The growing field of personalized medicine calls for the tailoring of treatment regimens based on the molecular signatures specific to an individual.  Currently, genomic biomarkers such as somatic or germline mutations in cancer can suggest a patient response to a drug.  Hoadad...